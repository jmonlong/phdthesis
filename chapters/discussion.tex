\chapter{Discussion of Results and Implications}
\label{chap:disc}


\section*{Population-based approaches and whole-genome sequencing}
% Population-based approach improves sensitivity and resolution. Useful for many applications, cancer, disease, repeats.
The three chapters highlighted the usefulness of population-based approach to detect genomic variation.
The superior sensitivity made it possible to detect somatic variants in tumor samples, small and non-coding variants in epilepsy patients and variants in low-mappability regions in healthy individuals.
% The results and implications that stemmed from these variants are described below.
The main results and implications concerning ccRCC, epilepsy and repeat-rich regions have been described in their respective chapters.
However, all these results relied on a first step of variant detection that used the same strategy: analyzing a genome in the context of many to minimize the effect of technical noise. 
This population-based approach can benefit variant calling as long as the genomes in the population are sequenced with similar protocol and machine, and the raw data is pre-processed with the same pipeline.
If this is the case, other factors that don't affect technical variation, for example gender or ethnicity, should not affect the variant calling.
The main result of this work, at least in term of methodology, is the demonstrated power of population-based approach across multiple applications.

All three chapters used WGS data across dozens or hundreds of genomes.
At the beginning of this work, WGS data across that many samples was not widespread.
Large-scale sequencing projects used to be rare and international efforts involving numerous centers\cite{Lander2001,Durbin2010}.
Yet, the cost of sequencing has been steadily decreasing and new machines currently advertise sub-1000\$ costs for whole-genome sequencing.
As a result, sequencing projects involving hundreds of genomes are more and more common.
Motivated by the future opportunities for personalized medicine, hundreds of genomes are being sequenced to characterize the population in different countries\cite{Francioli2014,Wong2013,Gudbjartsson2015}.
For disease studies as well, whole genome sequencing of hundreds of participants has been performed or is currently under way.
Large-scale project that focus on single disease include the MSSNG collaboration\footnote{\url{https://www.mss.ng}} that aims at sequencing 10,000 families affected by autism and whose first results were published recently\cite{Yuen2017} and the Alzheimer's Disease Sequencing Project\cite{Beecham2017} which is sequencing more than a thousand patients with Alzheimer disease.
WGS is also at the core of multi-disease projects that are currently in progress.
The Genomics England initiative\footnote{\url{https://www.genomicsengland.co.uk}} will sequence 100,000 genomes from patients with rare diseases and cancers while the TOPMed consortium\footnote{\url{https://www.nhlbiwgs.org/}} plans on sequencing the genome of 120,000 individuals to study the contribution of genetics to heart, lung, blood and sleep disorders.
Often, this data is first processed at the individual level and later pooled to derive population information or association metrics.
Our data highlights the benefit of pooling individuals earlier in the analysis workflow.
Instead of pooling results after variant calling, we found that variant calling could be significantly improved thanks to population information.
I believe that population-based approaches similar to {\sf PopSV} could have a large impact on the variant discovery across these large-scale projects.
It is regrettable to have information across hundreds of other experiments and to keep it aside for variant calling. 
Of note, promising avenues are currently being explored to reverse this trend, for example by integrating population information even before variant calling.
The new field of genome graphs aims at constructing a reference genome that includes known variation in order to directly improve read mapping, variant calling and population representation.

%% WGS
In addition to the population-based approach, our results also contribute to the case for whole-genome sequencing when designing a large scale genomic study.
Although more costly than genotyping arrays or exome sequencing, WGS can detect a larger range of variants: rare and frequent, coding and non-coding, SNVs and CNVs or other SVs.
WGS data produced with a primary objective in mind can often be re-used to study different aspects of genomic variation.
For example, the data used in Chapter \ref{chap:loy} was originally produced to describe somatic SNVs and broad CNV patterns in ccRCC\cite{Scelo2014}.
In Chapter \ref{chap:loy}, we re-analyzed this dataset to further investigate the arm-level aberrations across the cohort and more precisely estimate the amount of somatic LOY in tumors from male patients.
These results were replicated with a PCR-based approach in a different cohort of tumors and represent the foundation of the functional importance of somatic LOY advocated in our study.
In chapter \ref{chap:epi} and \ref{chap:rep}, the combination of WGS and {\sf PopSV} was also instrumental in discovering novel scientific results.
Novel coding variants in epilepsy or low-mappability regions were identified.
A CNV profile that had never been associated with epilepsy, rare non-coding CNVs, was for the first time seen to be strongly enriched close to known epilepsy genes.
Thousands of low-mappability regions that frequently experience CNV were identified, including some located near or within protein-coding genes.

\section*{Somatic loss of Y and gender imbalance in cancer}

The gender imbalance in the renal cancer incidence is not completely understood.
Our results suggest that somatic LOY occurs frequently in male tumors and have a functional impact.
Hence, loss of chromosome Y could explain part of the higher incidence in males.
Other cancers, such as liver and bladder cancers, show gender imbalance toward males and might also be partly explained by LOY\cite{Siegel2017}.
Several studies have found that more than 30\% of tumors from these cancers experienced LOY\cite{Sauter1995,Park2006}.
Our study and others suggest that LOY could be a driver mechanism of tumor progression by disregulating tumor suppressors, such as {\it KDM5D} and {\it KDM6C}.
Of note, {\it KDM5D} and {\it KDM6C} are both expressed in the liver and the bladder.
Because of the multiple similarities with our study of ccRCC, it is sensible to speculate that the same mechanism, that is downregulation of {\it KDM5D} and {\it KDM6C} through LOY, is involved in the tumor progression of cancers of the liver and bladder.
A population-based approach such as the one we used might identify more subtle LOY in sequencing datasets which could help estimate the rate and impact of this type of variation.

Although incidental, we detected LOY in the blood samples as well, both in the WGS and the PCR-based replication.
As described in the literature, somatic LOY in blood was associated with older age in our cohort.
Two studies suggest that the presence of LOY in blood samples can be associated with higher incidence rate of non-hematological cancers or Alzheimer disease\cite{Forsberg2014,Dumanski2016}.
Unfortunately, due to the absence of matched healthy controls we couldn't directly test the association with ccRCC.
We found no significant associations between LOY in blood and the tumor stage or grade.
Further investigations in the future will require more samples, including matched controls and balanced numbers of different tumor grades.


\section*{Small and non-coding CNVs in epilepsy and neurological disorders}

Our results suggest that WGS will be necessary to fully characterize the genetic factors associated with epilepsy.
First, WGS is more suitable for CNV detection compared to other sequencing approaches (e.g. exome sequencing).
Despite its successes in detecting rare SNVs associated with genetic disorders, exome sequencing is less suited for CNV and SV detection.
Indeed, the step that captures the regions of interest adds another layer of technical bias.
Not only is the read coverage affected by the capture efficiency in each region, the fragmented representation of the genome also hinders the sensitive detection of CNVs.
Considering the importance of CNVs in epilepsy as shown by our study and others, WGS will be key to efficiently detect even exonic variants associated with the disease.
Second, we show for the first time an association between non-coding CNVs and epilepsy.
Thanks to WGS, we were able to interrogate the presence of both small and non-coding variants.
In contrast, previous array-based study were limited to large CNVs which tended to overlap exonic regions.
We found a clear enrichment of rare non-coding CNVs in patients close to genes that were previously associated with epilepsy.
Even more convincing, the enrichment increased the closer to the exon and was boosted for deletions and variants overlapping regulatory regions.
Similar results were found in individuals with autism where non-coding {\it de novo} CNVs and SNVs were enriched up to 100 Kbp from autism-associated genes\cite{Turner2016}.

These conclusions are relevant for epilepsy but also for other neurological disorders.
Large CNVs have also been associated with autism\cite{Pinto2010}, mental retardation\cite{Mefford2009} and schizophrenia\cite{Walsh2008,Marshall2016}.
\begin{comment}
  Genes in neurological pathways tend to be large and organized in large gene families often located within segmental duplications. 
\end{comment}
In each disease, the same pattern emerged: probands tend to have rare large CNVs and/or in hotspot regions flanked by segmental duplications.
Often, the same CNV hotspot regions have been associated with several neurological disorders as is the case for 1q21.1, 15q13.3 and 16p11.2\cite{Mefford2009}.
Based on these commonalities and our results, I expect that small and non-coding CNVs play a role in other neurological disorders as well.
In general, the study of the genetics of neurological disorder could greatly benefit from WGS approaches similar to the one we used in chapter \ref{chap:epi}.

The enhanced resolution of WGS and our population-based approach suggest that some types of variants had been missed before.
The inclusion of such genomic variants might help characterize the exact syndrome or grouping patients with similar etiology.
Drug resistance is an important problem in idiopathic epilepsy and much remains to be done to understand its causes.
The study presented in chapter \ref{chap:epi} contributes to the identification of genes that might be associated with epilepsy and advocates for the inclusion of small and non-coding CNVs in the genomic profile of epilepsy patients.
It is my hope that either these candidate genes or the inclusion of those variants will one day help to devise new drugs or to assist clinician when choosing a treatment.


\section*{Including low-mappability regions in genome-wide studies}

More and more evidence supports the importance of repeated regions in disease and human phenotypes.
After being considered junk DNA for a long period, the role of many types of repeats is becoming clear.
From the highly variable STRs and their association with gene expression changes\cite{Gymrek2016}, to the contribution of TEs in regulation networks\cite{Bourque2009} or the satellite instability in cancer\cite{Kim2013}, repeats are gathering more attention.
Still, repeats are under-studied because of the technical challenges and reluctant mentality shift.
Our results show that WGS and a population-based approach is sufficient to detect CNV in many repeat-rich regions.
Moreover, we found that these regions are more likely to harbor CNVs compared to the rest of the genome.
Many low-mappability regions are also located close to protein-coding genes, some of which have been linked to Mendelian diseases before.
By masking repeats, genome-wide association studies have discarded a small part of the genome but a large fraction of the genomic variation.
I believe that approaches similar to the one presented in this work, as well as future technologies that will make repeat integration easier, could lead to a better genomic characterization of diseases and human phenotypes.




\chapter{Conclusions and Future Directions}
\label{chap:conc}

The population-based approaches presented here have been successful at detecting somatic and challenging germline changes in copy number but more remains to be done.
Although we were able to detect the presence of variation, more effort will be necessary to fully characterize the alternate alleles.
The methods could also be extended to other sequencing technologies, particularly targeted sequencing such as whole-exome sequencing.
Similar approaches could also be extended to other types of SVs, for example translocations and inversions.
Finally, recent technological advances shine a promising light on the future of SV characterization.
Genome editing might help investigate the functional impact of non-coding CNVs while new sequencing technologies will likely be used in concert to integrate SVs and low-mappability in genomics studies.

\section*{From variant detection to base-pair resolution}
% Long reads are the shit, but expensive shit.
The population-based approach presented in this work is powerful to detect variation but cannot fully characterize the variants.
Deriving the exact sequence of the mutated allele or breakpoints remains challenging in repeat-rich regions or for complex SVs.
Indeed, large sample sizes and population-oriented analysis are mostly able to flag the presence of an abnormal signal in these regions.
To unlock a base-pair level resolution for the breakpoints or the variant sequence in general, a different type of data will be necessary rather than more of the same data.
In WGS early years, genomes were sequenced at low depth and pooling individuals helped increase read support necessary to estimate the variant sequences.
A typical WGS genome is now often 30x deep or more and pooling reads across different individuals will have only marginal benefit on the base-pair resolution of the SVs.
At this depth, the problematic variants are either in repeat-rich regions or more complex than the canonical SV types.
More of the same short reads wouldn't bring much new information and the ambiguity would likely remain.
In the end, the short read size is the main limitation.
Hence, long read sequencing will likely have a large impact on characterizing many of the SVs that remain challenging.
Several studies already showed the power of long reads for SV detection\cite{Chaisson2014,Pendleton2015}.
Unfortunately the higher cost of these technologies limits their use.
The cost and usage might change but it is unlikely that a very large number of genome will be sequenced with these technologies.
Still, this technology could be used to validate and confirm SVs identified in large scale short-read projects.
By carefully choosing a few genomes to sequence, the catalog of complex and repeat-rich SVs could be greatly improved.
For example, sequencing genomes carrying candidate pathogenic variants could validate and provide more insights into the variant impact or potential mechanism of action.
Selecting a few genomes with different complex SVs and low-mappability variants could also maximize the gain from each long-read sequencing experiment.
In summary, long-read sequencing of a few carefully selected genomes would nicely complement deep short-read sequencing across large cohorts.
Of note, a more cost effective approach would be to capture the regions containing the SVs detected from the large-scale short-read surveys.
The challenge here is to efficiently and accurately capture large and potentially repeat-rich regions.
If this type of capture is possible, Sanger sequencing could also be used to characterize the variants at the base-pair level.
Having a validated set of variants at the base-pair level and covering low-mappability regions will also provide an extremely useful gold-standard to assess the sensitivity and specificity of CNV methods.
In the absence of such a gold-standard, we had to rely on indirect assessment, such as the replication in twins or the comparison with just a few long-read assemblies.

\section*{Extension to targeted sequencing}
% Other application WES/targeted sequencing
Targeted sequencing involve a step of capture in which the desired genomic regions are selectively amplified.
Thanks to the capture, only a set of regions, for example coding regions, are sequenced.
However, the capture efficiency varies depending on the design and regions, introducing additional technical variation in the read coverage.
CNV detection from the read coverage is challenging as a result.
Existing methods turned to population-based approach, similar to our WGS method {\sf PopSV}, in order to normalize read coverage using other experiments that used similar capture\cite{Shi2013,Talevich2014}.
{\sf PopSV} could be easily extended to such application as it doesn't rely on coverage uniformity but rather considers each region separately in the context of reference samples.
We actually ran {\sf PopSV} successfully on several whole-exome sequencing projects with minimal changes.
Briefly, one additional step in the pipeline was necessary: the removal of regions that were not covered by the sequencing experiment.
Of note, the quality control step was even more important here in order to ensure that the reference and tested samples originated from the same sequencing protocol.
Indeed, capture protocols are continuously evolving with new exome capture kits available every year.
Although we believe that the normalization used by {\sf PopSV} is superior to several of existing methods, the regions with abnormal coverage are merged using a simplistic approach.
For WGS, it is natural to merge consecutive regions that share a CNV signal as they are located one next to the other.
In targeted sequencing, there are often stretches of non-covered regions that separate captured regions.
By integrating this information a better segmentation can be performed as shown by several methods\cite{Fromer2012,Magi2013}.
In the context of an extension to targeted sequencing, we could improve {\sf PopSV}'s segmentation step with such algorithms in order to have both a more advanced normalization and an appropriate segmentation.

\section*{Extension to balanced structural variation}
% Other application: other types of SVs (balanced)
{\sf PopSV} could also be extended to other types of SVs, as the name originally intended.
Currently, abnormal coverage of properly mapped reads identifies regions with a change in copy number.
In the future, {\sf PopSV} could operate on the coverage of reads with discordant paired-end mapping.
For example, reads whose pair couldn't be aligned or aligned at an abnormal distance or orientation.
An abnormal excess of discordant reads would be a sign of SV.
With {\sf PopSV}'s population-based approach, discordant mapping due to repeats will be corrected for and only regions with a real excess of discordant mapping should be called.
This type of test could be run across genomic regions, as for the CNV analysis, or between pairs genomic regions.
The objective of the paired-regions approach is to increase the signal-to-noise ratio between the SV-caused discordant reads and the background level.
Indeed, the background level of discordant reads in a region might be high enough to drown the signal from the additional reads created by the SV.
When focusing on discordant read pairs with one read mapped to a region and the other in another distant region, the amount of background discordant mapping is much lower.
The excess of discordant read pair linking the boundaries of a translocation or an inversion might be more easily detected.
Both of these approaches are currently being tested.
The original scan, i.e. testing one region at a time, seems to be under-powered and return much fewer calls than the CNV scan.
Alternatively, this extension of {\sf PopSV} would also be useful to annotate calls from other methods, for example those that identify clusters of discordant or split reads.
These methods tend to suffer from a high false-positive rate most likely due to repeat confusion.

\section*{Investigating the functional impact of non-coding CNVs}

We identified dozens of rare non-coding CNVs located close to known epilepsy genes and absent from our controls and public CNV databases.
The most promising candidates are located in regions that were previously associated with changes in the gene expression and are predicted to host regulatory regions.
While the frequency and location of these non-coding CNVs are encouraging, it is not clear which variant really has an impact on the gene and eventually the disease phenotype.
Indeed, we observed a clear enrichment which means that some CNVs are associated with the disease.
In order to narrow down the list of candidate regions we could either increase the sample size or experimentally test the variant's impact.
Both strategies have their own set of challenges.

To achieve a higher sample size, more patients are necessary and might require national or international collaboration.
Because epilepsy is a diverse disease, the new patients should be matched as much as possible with our cohort in order to maximize the chances of observing recurrent variants.
Finally, the number of probands that need to be sequenced to identify single genes or non-coding regions might be unrealistically high, again because of the diverse phenotype and complex gene network involved in the disease mechanism.

Investigating deeper our current candidates might be more feasible thanks to recent advances in cell reprogramming and DNA editing.
It is not feasible to study live human brains and unpractical to collect post-mortem brains in correct conditions for functional assays.
Cell reprogramming provides a better alternative.
By reprogramming cells from a carrier into relevant cell types, e.g. neurons, the effect of a variant on the gene function could be tested in the laboratory.
If patients' samples are unavailable or to better control for the genetic background, DNA editing like CRISPR-Cas9 could recreate the non-coding deletion in cells.
DNA editing before cell differentiation would provide a culture of cells that could then be differentiated into different cell types to extensively study the effect of the variants.
While a true ``epilepsy'' phenotype is almost impossible to test in a cell culture, we could investigate the effect of the variants on gene expression or other molecular phenotypes.
Although this approach requires time, resources and expertise, it is more geared toward validating our non-coding candidates and more likely to give conclusive results.

\section*{SV and WGS in the near future}

Thanks to better sequencing technologies and methods, systematic {\it de novo} assembly of genomes might eventually supplant the re-sequencing approach that is dominant today.
The optimal output would be phased sequence for each genome that is sequenced and assembled.
SV detection would then come down to comparing the assembled genomes.
Recent advances in long-read, linked-read, or conformation capture sequencing greatly improve the quality of the {\it de novo} assemblies and contains long-range information useful for phasing.
However, for SV detection, long-read sequencing will be necessary to reach a base-pair resolution across the full genome, as mentioned above.
Because of its cost, only a limited number of genomes will be sequenced with long-read sequencing in the near future.
Furthermore, hundreds of thousands of genomes have been sequenced or are being sequenced with short-read technology.
Until affordable and comprehensive {\it de novo} assembly is available, the field will likely move to hybrid strategies for SV analysis.
But how could we integrate SV information from different approaches and across hundreds to thousands of genomes?
The recent development of genome graphs provides a promising solution\cite{Paten2017}.
Genome graphs represent the genome and genomic variation in a population using a graph structure.
One of its current form corresponds to the current reference genome augmented with SNVs and indels from the 1000 Genomes Project\cite{Garrison2017}.
Genome graphs are by nature flexible so that we could imagine further improving their breadth with high-quality {\it de novo} assemblies or SV catalogs from long-read sequencing datasets.
Using a genome graph populated with the high-resolution variants, SV could be more efficiently genotyped in short-reads datasets across large populations.
Genome graphs also provide an ideal structure to represent complex haplotypes, such as those involving SVs.
As for other types of variants, haplotype information will be invaluable to assist variant calling and to predict the functional impact of SVs.
Short-read datasets could benefit from the long-range information present in the high-resolution datasets by integrating complex haplotype information in the genome graphs.
With coordination and data sharing, and with genome graphs as a new reference system, there is hope that we can rapidly reach the point where phased high-resolution SVs can be genotyped accurately from short-read datasets spanning hundreds of thousands of genomes.




%%% Local Variables:
%%% mode: latex
%%% TeX-master: "../main"
%%% End:
